\documentclass[journal,12pt,twocolumn]{IEEEtran}
%
\usepackage{setspace}
\usepackage{gensymb}
%\doublespacing
\singlespacing

%\usepackage{graphicx}
%\usepackage{amssymb}
%\usepackage{relsize}
\usepackage[cmex10]{amsmath}
%\usepackage{amsthm}
%\interdisplaylinepenalty=2500
%\savesymbol{iint}
%\usepackage{txfonts}
%\restoresymbol{TXF}{iint}
%\usepackage{wasysym}
\usepackage{amsthm}
%\usepackage{iithtlc}
\usepackage{mathrsfs}
\usepackage{txfonts}
\usepackage{stfloats}
\usepackage{bm}
\usepackage{cite}
\usepackage{cases}
\usepackage{subfig}
%\usepackage{xtab}
\usepackage{longtable}
\usepackage{multirow}
%\usepackage{algorithm}
%\usepackage{algpseudocode}
\usepackage{enumitem}
\usepackage{mathtools}
\usepackage{steinmetz}
\usepackage{tikz}
\usepackage{circuitikz}
\usepackage{verbatim}
\usepackage{tfrupee}
\usepackage[breaklinks=true]{hyperref}
%\usepackage{stmaryrd}
\usepackage{tkz-euclide} % loads  TikZ and tkz-base
%\usetkzobj{all}
\usetikzlibrary{calc,math}
\usepackage{listings}
    \usepackage{color}                                            %%
    \usepackage{array}                                            %%
    \usepackage{longtable}                                        %%
    \usepackage{calc}                                             %%
    \usepackage{multirow}                                         %%
    \usepackage{hhline}                                           %%
    \usepackage{ifthen}                                           %%
  %optionally (for landscape tables embedded in another document): %%
    \usepackage{lscape}     
\usepackage{multicol}
\usepackage{chngcntr}
%\usepackage{enumerate}

%\usepackage{wasysym}
%\newcounter{MYtempeqncnt}
\DeclareMathOperator*{\Res}{Res}
%\renewcommand{\baselinestretch}{2}
\renewcommand\thesection{\arabic{section}}
\renewcommand\thesubsection{\thesection.\arabic{subsection}}
\renewcommand\thesubsubsection{\thesubsection.\arabic{subsubsection}}

\renewcommand\thesectiondis{\arabic{section}}
\renewcommand\thesubsectiondis{\thesectiondis.\arabic{subsection}}
\renewcommand\thesubsubsectiondis{\thesubsectiondis.\arabic{subsubsection}}

% correct bad hyphenation here
\hyphenation{op-tical net-works semi-conduc-tor}
\def\inputGnumericTable{}                                 %%

\lstset{
%language=C,
frame=single, 
breaklines=true,
columns=fullflexible
}
%\lstset{
%language=tex,
%frame=single, 
%breaklines=true
%}

\begin{document}
%


\newtheorem{theorem}{Theorem}[section]
\newtheorem{problem}{Problem}
\newtheorem{proposition}{Proposition}[section]
\newtheorem{lemma}{Lemma}[section]
\newtheorem{corollary}[theorem]{Corollary}
\newtheorem{example}{Example}[section]
\newtheorem{definition}[problem]{Definition}
%\newtheorem{thm}{Theorem}[section] 
%\newtheorem{defn}[thm]{Definition}
%\newtheorem{algorithm}{Algorithm}[section]
%\newtheorem{cor}{Corollary}
\newcommand{\BEQA}{\begin{eqnarray}}
\newcommand{\EEQA}{\end{eqnarray}}
\newcommand{\define}{\stackrel{\triangle}{=}}

\bibliographystyle{IEEEtran}
%\bibliographystyle{ieeetr}


\providecommand{\mbf}{\mathbf}
\providecommand{\pr}[1]{\ensuremath{\Pr\left(#1\right)}}
\providecommand{\qfunc}[1]{\ensuremath{Q\left(#1\right)}}
\providecommand{\sbrak}[1]{\ensuremath{{}\left[#1\right]}}
\providecommand{\lsbrak}[1]{\ensuremath{{}\left[#1\right.}}
\providecommand{\rsbrak}[1]{\ensuremath{{}\left.#1\right]}}
\providecommand{\brak}[1]{\ensuremath{\left(#1\right)}}
\providecommand{\lbrak}[1]{\ensuremath{\left(#1\right.}}
\providecommand{\rbrak}[1]{\ensuremath{\left.#1\right)}}
\providecommand{\cbrak}[1]{\ensuremath{\left\{#1\right\}}}
\providecommand{\lcbrak}[1]{\ensuremath{\left\{#1\right.}}
\providecommand{\rcbrak}[1]{\ensuremath{\left.#1\right\}}}
\theoremstyle{remark}
\newtheorem{rem}{Remark}
\newcommand{\sgn}{\mathop{\mathrm{sgn}}}
\providecommand{\abs}[1]{\left\vert#1\right\vert}
\providecommand{\res}[1]{\Res\displaylimits_{#1}} 
\providecommand{\norm}[1]{\left\lVert#1\right\rVert}
%\providecommand{\norm}[1]{\lVert#1\rVert}
\providecommand{\mtx}[1]{\mathbf{#1}}
\providecommand{\mean}[1]{E\left[ #1 \right]}
\providecommand{\fourier}{\overset{\mathcal{F}}{ \rightleftharpoons}}
%\providecommand{\hilbert}{\overset{\mathcal{H}}{ \rightleftharpoons}}
\providecommand{\system}{\overset{\mathcal{H}}{ \longleftrightarrow}}
	%\newcommand{\solution}[2]{\textbf{Solution:}{#1}}
\newcommand{\solution}{\noindent \textbf{Solution: }}
\newcommand{\cosec}{\,\text{cosec}\,}
\providecommand{\dec}[2]{\ensuremath{\overset{#1}{\underset{#2}{\gtrless}}}}
\newcommand{\myvec}[1]{\ensuremath{\begin{pmatrix}#1\end{pmatrix}}}
\newcommand{\mydet}[1]{\ensuremath{\begin{vmatrix}#1\end{vmatrix}}}
%\numberwithin{equation}{section}
\numberwithin{equation}{subsection}
%\numberwithin{problem}{section}
%\numberwithin{definition}{section}
\makeatletter
\@addtoreset{figure}{problem}
\makeatother

\let\StandardTheFigure\thefigure
\let\vec\mathbf
%\renewcommand{\thefigure}{\theproblem.\arabic{figure}}
\renewcommand{\thefigure}{\theproblem}
%\setlist[enumerate,1]{before=\renewcommand\theequation{\theenumi.\arabic{equation}}
%\counterwithin{equation}{enumi}


%\renewcommand{\theequation}{\arabic{subsection}.\arabic{equation}}

\def\putbox#1#2#3{\makebox[0in][l]{\makebox[#1][l]{}\raisebox{\baselineskip}[0in][0in]{\raisebox{#2}[0in][0in]{#3}}}}
     \def\rightbox#1{\makebox[0in][r]{#1}}
     \def\centbox#1{\makebox[0in]{#1}}
     \def\topbox#1{\raisebox{-\baselineskip}[0in][0in]{#1}}
     \def\midbox#1{\raisebox{-0.5\baselineskip}[0in][0in]{#1}}

\vspace{3cm}


\title{Assignment 8}
\author{Jayati Dutta}





% make the title area
\maketitle

\newpage

%\tableofcontents

\bigskip

\renewcommand{\thefigure}{\theenumi}
\renewcommand{\thetable}{\theenumi}
%\renewcommand{\theequation}{\theenumi}


\begin{abstract}
This is a simple document explaining how to determine the nullspace solution from the Row Reduced Echelon Form of a coefficient matrix.
\end{abstract}

%Download all python codes 
%
%\begin{lstlisting}
%svn co https://github.com/JayatiD93/trunk/My_solution_design/codes
%\end{lstlisting}

Download all and latex-tikz codes from 
%
\begin{lstlisting}
svn co https://github.com/gadepall/school/trunk/ncert/geometry/figs
\end{lstlisting}
%


\section{Problem}
Find all solutions to the following system of equations by row-reducing the co-efficient matrix:
\begin{align}
\frac{1}{3}x_1 +2x_2 - 6x_3 =0\\
-4x_1\quad \quad+ 5x_3=0\\
-3x_1+6x_2-13x_3=0\\
-\frac{7}{3}x_1 +2x_2 - \frac{8}{3}x_3 =0
\end{align}


\section{Explanation}
The coefficient matrix is:
\begin{align}
A=\myvec{\frac{1}{3} & 2 & -6\\-4 & 0 & 5\\-3 & 6 & -13\\-\frac{7}{3} & 2 & -\frac{8}{3}}
\end{align}
The number of rows of this coefficient matrix is $m$ = 4 and the number of columns is $n$= 3, So in this case, $n<m$.
Now the row operations are:
\begin{align}
\myvec{\frac{1}{3} & 2 & -6\\-4 & 0 & 5\\-3 & 6 & -13\\-\frac{7}{3} & 2 & -\frac{8}{3}}\xleftrightarrow[R_1\leftarrow R_1\times3]{R_4\leftarrow R_4\times3}
\myvec{1 & 6 & -18\\-4 & 0 & 5\\-3 & 6 & -13\\-7 & 6 & -8}\\
\xleftrightarrow[]{R_3\leftarrow R_2+R_3}
\myvec{1 & 6 & -18\\-4 & 0 & 5\\-7 & 6 & -8\\-7 & 6 & -8}\xleftrightarrow[]{R_4\leftarrow R_4-R_3}\\
\myvec{1 & 6 & -18\\-4 & 0 & 5\\-7 & 6 & -8\\0 & 0 & 0}\xleftrightarrow[R_3\leftarrow R_3+7R_1]{R_2\leftarrow R_2+4R_1}
\myvec{1 & 6 & -18\\0 & 24 & -67\\0 & 48 & -138\\0 & 0 & 0}\\
\xleftrightarrow[]{R_3\leftarrow R_3/2}\myvec{1 & 6 & -18\\0 & 24 & -67\\0 & 24 & -69\\0 & 0 & 0}\xleftrightarrow[]{R_3\leftarrow R_3-R_2}\\
\myvec{1 & 6 & -18\\0 & 24 & -67\\0 & 0 & -2\\0 & 0 & 0}\xleftrightarrow[]{R_3\leftarrow \frac{R_3}{(-2)}}
\myvec{1 & 6 & -18\\0 & 24 & -67\\0 & 0 & 1\\0 & 0 & 0}\\
\xleftrightarrow[]{R_2\leftarrow \frac{R_2}{4}}
\myvec{1 & 6 & -18\\0 & 6 & -\frac{67}{4}\\0 & 0 & 1\\0 & 0 & 0}\xleftrightarrow[]{R_1\leftarrow R_1-R_2}\\
\myvec{1 & 0 & -\frac{5}{4}\\0 & 6 & -\frac{67}{4}\\0 & 0 & 1\\0 & 0 & 0}\xleftrightarrow[]{R_2\leftarrow \frac{R_2}{6}}
\myvec{1 & 0 & -\frac{5}{4}\\0 & 1 & -\frac{67}{24}\\0 & 0 & 1\\0 & 0 & 0}\\
\xleftrightarrow[R_1\leftarrow R_1+\frac{5R_3}{4}]{R_2\leftarrow R_2+\frac{67R_3}{24}}
\myvec{1 & 0 & 0\\0 & 1 & 0\\0 & 0 & 1\\0 & 0 & 0}
\end{align}
Now,
\begin{align}
A\vec{x} = \myvec{1 & 0 & 0\\0 & 1 & 0\\0 & 0 & 1\\0 & 0 & 0}\myvec{x_1\\x_2\\x_3} = 0
\end{align}

%Now the rank of the matrix $A$ is $\rho$ = 3 =$n$, so the coefficient matrix has the full column rank. $\rho$ = 3 =$n$,this implies that the solution may or may not exist. But as $\rho = n < m$, so if solution exists then it will be either 1 or 0.

So,
\begin{align}
1 x_1 + 0 x_2 + 0 x_3 = 0\\
0 x_1 + 1 x_2 + 0 x_3 = 0\\
0 x_1 + 0 x_2 + 1 x_3 = 0
\end{align}
The coefficient matrix for these equations is: $\myvec{1 & 0 & 0\\0 & 1 & 0\\0 & 0 & 1}$ = $I_3$ and it is invertible.

As the coefficient matrix is invertible , so
\begin{align}
\vec{I_3} \vec{x} = \vec{0}\\
\implies \vec{x}= \vec{0}
\end{align} 
%\renewcommand{\theequation}{\theenumi}
%\begin{enumerate}[label=\thesection.\arabic*.,ref=\thesection.\theenumi]
%\numberwithin{equation}{enumi}
%\item Verification of the above problem using python code.\\
%\solution The  following Python code generates Fig. \ref{fig:hyperbola}
%\begin{lstlisting}
%codes/hyperbola_3.py
%\end{lstlisting}
%%
%\end{enumerate}

\end{document}



